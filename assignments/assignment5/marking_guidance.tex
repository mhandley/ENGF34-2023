\documentclass{article}

\usepackage{graphicx}
\usepackage{amsmath}    
        
\title{Design \& Professional Skills --- Pacman Protocol Specification Assignment: Marking Guidance}     
 
\author{}
\date{}
        
\begin{document}

\maketitle

\section*{Motivation}

Why are we getting you to mark each other's coursework?  The aim here
is that you have tried hard to write a correct, complete and
unambiguous specification.  That is a valuable skill for a computer
scientist, but you only really learn whether a specification is
sufficient when you try to implement someone else's.  Thus we want you
to read and critique each other's specification.  Ideally your written
feedback would be sufficient to help your classmate remedy a
deficiency in their specification, thus you learn by trying to
interpret a specification that is not your own, and they learn where
their efforts fell short from the viewpoint of someone who might have
to code from their implementation.  This only works well though if
everyone puts in effort trying to constructively critique each other's
specifications.

\subsection*{General Guidance}

As stated in the assignment sheet, the main criteria for marking are:
\begin{itemize}
  \item Conciseness.  Don't waffle.  Be specific.
  \item Correctness.  Will the protocol fail if implemented as specified?
  \item Unambiguous.  Do you understand how to code what is specified in all cases?
  \item Completeness. Are some things missing?
\end{itemize}

The primary goal of a specification is that someone else should be able to
code an implementation of the protocol that will interoperate with
other implementations.

Broadly, you should assign marks for Correctness, Unambiguousness (is
that a word?), and Completeness.

Conciseness is not a primary marking criterion, but you may deduct one
mark (out of 30 total) for excessive waffle that does not contribute
to an understanding of the protocol.  Please do not deduct marks for
lack of conciseness unless it really is excessive.

\subsection*{Correctness}

You are assigning marks regarding whether you believe the protocol
will perform the task assigned correctly.  A protocol can be incorrect for many reasons.  For example:
\begin{itemize}
 \item it may omit sending information that is needed
\item if may encode information in a way that loses information when that is not acceptable (sometimes losing information can be acceptable, but any protocol that deliberately does so should explain why, or you can consider this to be unintentional).
\item it may fail to behave correctly under some circumstances.  This is more likely with UDP-based protocols that fail to correctly handle packet loss.
\end{itemize}
Generally, you're reading to see if you can see any obvious flaws. You are not expected to need to implement the protocol to spot such flaws though.

\paragraph{Asssign a mark out of 10 for correctness}.

Rough guidance:
\begin{itemize}
\item 10: I can see no error in this protocol, and it looks like it will work in all circumstances I can see.
\item 7:  (threshold for an A grade)  It looks like there's a minor error, but it would be easily corrected.
\item 5:  (threshold for a C grade)  Protocol contains a few errors, but most of it is there.
\item 4: (throughhold for a pass)  Protocol contains quite a few errors, but there's a reasonable attempt at a design.

\item 0: Not enough design submitted to be able to determine correctness.
\end{itemize}

\subsection*{Unambiguousness}

For this section, don't concern yourself with parts of the protocol
that have not been specified, or whether the parts that are specified
are correct.  What you are attempting to do is to decide whether you
could unambiguously implement the specified protocol (even if the
resulting implementation has correctness errors or is incomplete).

\begin{itemize}
\item Do you know when each message should be sent?  If it's not explicit, but is obvious, that's not a problem.  

\item When a message is received, do you know how to decode it using
software?

\item Is it clear how to separate a TCP bytestream into messages?
This may not be explicitly written, but if it is obvious how to do it,
such as the fixed sized messages in pong v1 identified by the type
field, then that is not ambiguous.

\item With a text-based encoding, could you write a parser that can
tell a legal message from an illegal one? 
\end{itemize}

Sometimes the specification may not be perfect, but there is an
example that makes clear what was intended.  In such cases, do not
mark down if the example is sufficient to disambiguate.

\paragraph{Assign a mark out of 10 for umambiguousness.}

Rough guidance:

\begin{itemize}
\item 10:  I know exactly how to decode all the messages, how to tell a legal message from an illegal message, when I should send each message, and what to do about packet loss (UDP only).

\item 7:  This is a good and largely unambiguous specification, but falls slightly short on one or two points.  I could implement it, but I wouldn't be 100\% sure I'd done it the same way someone else would.

\item 5:  This is an OK protocol, but I couldn't be sure how to implement it without asking quite a few clarifying questions.

\item 4:  I generally understand roughly what is intended, but I don't known unambiguously how to interpret most of the messages.

\item 0: There's too little protocol here to determine if its unambiguous.
\end{itemize}

\subsection*{Completeness}

Are there any gaps in what has been specified?  Are all the messages
specified?  Is it specified when to send them (if this not explicit,
but is obvious enough that you could implement a complete protocol, don't
mark down for it)?  If the protocol uses UDP, does it indicate how to
handle packet loss for each message type that uses UDP?

Generally, completeness is easier to assess than Correctness or
Unambiguousness, because gaps in completeness will be more obvious.

\paragraph{Assign a mark out of 10 for completeness.}

Rough guidance:

\begin{itemize}
\item 10:  Specification appears to be complete in every aspect.  I can't think of anything else that is necessary to add to achieve interoperability.

\item 7:  Specification is largely complete.  There are one or two ommissions, but I think I could largely implement a complete protocol despite the ommisssions.

\item 5:  There are significant gaps in the specification.  I could implement part of the protocol, but would need more information to complete it.

\item 4:  Part of the protocol is specified, but more than half is not.  I could not implement most of this protocol without more information.

\item 0:  There's no actual specification here - I couldn't implement any of the protocol without more information.
\end{itemize}

Again, if there is a gap in completeness, but an example explains well
enough that you could implement this part anyway, this may offset the
missing specification.

\subsection*{Overall Mark}

For each marking section (Correctness, Unambuguousness and
Completeness), you must write a short explanation of why you allocated
your mark.  Normally one paragraph will be sufficient, but you can
write more if you wish.  Try to be specific: what precisely is
missing; which message is ambiguous, and why; what is an example where
the protocol would behave incorrectly?

Try to write in a constructive style that would help your classmate
improve their specification.  If it helps, imagine you had to stand up
at the microphone in a large standards meeting to make your comment to
the document author, with the aim of improving the draft standard
document.  If a comment would not be appropriate in such a setting, it
is probably not appropriate as feedback here either.

Add together your marks out of 10 for correctness, unambigiousness,
and completeness to obtain a mark out of 30.

You may amend this mark if necessary, if you think it does not reflect
a fair grade for the specification.
\begin{itemize}
\item You may add up to 2 marks for
elegance of design, if you think the design is really cute in some
way, so long as this does not take the total mark above 30.
\item You may subtract up to 1 mark for lack of conciseness, as
  discussed in the intro to this guidance.  \textbf{Do not do this unless the
  specification contains very excessive waffle}.
\item You may subtract up to 5 marks for other flaws not listed, \textbf{but
you cannot do this unilaterally} - substraction of marks needs to be
clearly justified, and needs to be approved by me via a Piazza private
message. 
\end{itemize}

The final mark will be the median mark of the student marks for the
coursework, as this is generally more robust to outliers than the
mean.

If anyone feels the marks they received are not justified by the
feedback, there will be a process by which this can be contested.  If
marks are contested, the final mark will be allocated by the course
staff, but such contested marks will involve a re-mark of the whole
coursework, and so marks may move either up or down.

If any vindictive feedback is reported to us, the person who wrote
that feedback may themselves have marks deducted.

\end{document}
