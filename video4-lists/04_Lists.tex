\documentclass{beamer} % [aspectratio=169]
\usetheme{ucl}
\setbeamercolor{banner}{bg=darkred}
\setbeamersize{description width=2em}
\setbeamertemplate{navigation symbols}{\vspace{-2ex}} 

\usepackage[T1]{fontenc} % Turn £ into $
\usepackage{minted}
\usemintedstyle{emacs}

\usepackage{fancyvrb}
\usepackage{xcolor}
\usepackage{url}

\usepackage{natbib}
\usepackage{bibentry}
\usepackage{url}

\newenvironment{variableblock}[3]{%
  \setbeamercolor{block body}{#2}
  \setbeamercolor{block title}{#3}
  \begin{block}{#1}}{\end{block}}

\newcommand\emc[1]{\textcolor{midred}{\textbf{#1}}}

\AtBeginSection[]{
  \begin{frame}
  \vfill
  \centering
  \begin{beamercolorbox}[sep=8pt,center,shadow=true,rounded=true]{title}
    \usebeamerfont{title}\insertsectionhead\par%
  \end{beamercolorbox}
  \vfill
  \end{frame}
}

\author{Prof.\ Mark Handley \\ University College London, UK}
\title{Python Lists}
\subtitle{ENGF34: Design and Professional Skills }
% \institute{}
\date{Term 1, 2023}

\begin{document}
\nobibliography*

\frame{
\titlepage
}

\section{Data Structure of the Week}

\begin{frame}
  \frametitle{Python Lists}
  Often you need to store a bunch of data.  In python, one way to do this is to use a \emc{list}.
	\inputminted[
		xleftmargin=1.4em,
		%frame=lines,
		%framesep=2mm,
		%baselinestretch=1.2,
		bgcolor=stone,
		fontsize=\footnotesize,
		%linenos
	]{python}{src/interactive_lists.py}
  
\end{frame}

\begin{frame}
  \frametitle{Deleting and Inserting in Python Lists}
	\inputminted[
		xleftmargin=1.4em,
		%frame=lines,
		%framesep=2mm,
		%baselinestretch=1.2,
		bgcolor=stone,
		fontsize=\footnotesize,
		%linenos
	]{python}{src/interactive_lists2.py}
  
\end{frame}

\begin{frame}
  \frametitle{Be careful about copying Lists}
	\inputminted[
		xleftmargin=1.4em,
		%frame=lines,
		%framesep=2mm,
		%baselinestretch=1.2,
		bgcolor=stone,
		fontsize=\footnotesize,
		%linenos
	]{python}{src/interactive_lists3.py}
  
\end{frame}

\begin{frame}
  \frametitle{Iterating over Python Lists}
	\inputminted[
		xleftmargin=1.4em,
		%frame=lines,
		%framesep=2mm,
		%baselinestretch=1.2,
		bgcolor=stone,
		fontsize=\footnotesize,
		%linenos
	]{python}{src/interactive_lists4.py}
  
\end{frame}


\bibliographystyle{alpha}
\nobibliography{references}

\end{document}
